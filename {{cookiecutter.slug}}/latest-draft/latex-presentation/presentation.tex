\documentclass[aspectratio=169]{beamer}
% The page prefix can be used to put unique labels
% on each slide of a set of presentations.
\usetheme[pageprefix=demo\_]{distractionfree}

\definecolor{tab.blue}{HTML}{1f77b4}
\definecolor{tab.orange}{HTML}{ff7f0e}
\definecolor{tab.green}{HTML}{2ca02c}
\definecolor{tab.red}{HTML}{d62728}
\definecolor{tab.purple}{HTML}{9467bd}
\definecolor{tab.brown}{HTML}{8c564b}
\definecolor{tab.pink}{HTML}{e377c2}
\definecolor{tab.gray}{HTML}{7f7f7f}
\definecolor{tab.olive}{HTML}{bcbd22}
\definecolor{tab.cyan}{HTML}{17becf}

% This makes it possible to add backup slides, without counting them.
\usepackage{appendixnumberbeamer}
\renewcommand{\appendixname}{\texorpdfstring{\translate{appendix}}{appendix}}

\usepackage{hyperref}
\hypersetup{hidelinks=true,allcolors=uni.blue,linktoc=none}

%
% Title slide information
%

\title[Short title]{A more elaborate and informative version of the title}
\author{\underline{Presenting Autor}, Other authors \\
\texttt{\scriptsize first.last@email.com} \\}
\subtitle[Subtitle]{Optional subtitle}
\institute{%
Venue \\[0em]
Your institute\\[0em]
Logos \\[0em]
}
\date{\today}

\begin{document}

\maketitle

% For an actual presentations, you may want to use input commands
% to organize the slides over multiple files.

% Example of a fullscreen image.
% This type of slide can be used to focus the audience
% on some question, topic or on the presenter.
% Image adapted from https://www.pexels.com/photo/domesticated-horses-grazing-in-pasture-on-farmland-5838235/
{
\usebackgroundtemplate{\includegraphics[width=\paperwidth]{pexels-mathias-reding-5838235.jpg}}
\begin{frame}[plain]
\end{frame}
}

% A slide without title, again useful if you want to emphasize something
\begin{frame}[plain]
    \centering
    \begin{minipage}{9cm}
        \begin{center}
            \it\centering\large
            Education is not the learning of facts, \\
            but the training of the mind to think.
        \end{center}
        \flushright --- Albert Einstein, 1921 \qquad
    \end{minipage}
\end{frame}

\section{Section to organize slides in a long presentation}

% Ordinary slide with title.
% Slide titles can be used to organize, e.g. clarify that some slides belong together
% under some topic, but usually they are just a nuisance.
\begin{frame}{Do you really want a title on your slide?}
    Think about it:
    \begin{vfilleditems}
        \item
        More stuff to read will usually not always help the audience.
        \item
        You want the audience to listen, not to read.
        \item
        You definitely do not want the presenter to read.
    \end{vfilleditems}
    You have to admit: this slide has too much text for a presentation.
\end{frame}

% Occasionally, it is useful to have backup slides,
% e.g. to support answers to questions after the presentation.
% The appendix below shows how this can be done
% without increasing the total number of slides.

\appendix
\begin{frame}[plain]
\end{frame}
% TODO: make \section* work
\section{Backup slides}

\begin{frame}{Some more details}
    Details here
\end{frame}


\end{document}
